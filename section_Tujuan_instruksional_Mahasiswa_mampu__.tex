\section{Tujuan instruksional} 

Mahasiswa mampu memahami:
\begin{itemize}
\item Prinsip dasar uji pompa
\end{itemize}
\begin{itemize}
\item Bagaimana melakukan uji poma
\end{itemize}
\begin{itemize}
\item Mengambil data uji pompa
\end{itemize}


\section{Pendahuluan}

Uji akuifer (UA) adalah pengujian properti hidrolik akuifer melalui proses pemompaan (uji pompa), dengan debit tetap maupun debit berubah. Teknik ini diyakini menghasilkan estimasi properti hidrolik akuifer yang paling tepat, serta perhitungan debit sumur optimum atau sering disebut debit aman (\textit{safe yield}). Dengan uji akuifer, kita juga akan mendapatkan profil penurunan MAT dalam proses pemompaan. Dalam jangka panjang, uji akuifer juga diharapkan dapat berkontribusi dalam menata lingkungan secara berkelanjutan.


\subsection{Beberapa definisi}

An aquifer test (or a pumping test) is conducted to evaluate an aquifer by "stimulating" the aquifer through constant pumping, and observing the aquifer's "response" (drawdown) in observation wells. Aquifer testing is a common tool that hydrogeologists use to characterize a system of aquifers, aquitards and flow system boundaries.

A slug test is a variation on the typical aquifer test where an instantaneous change (increase or decrease) is made, and the effects are observed in the same well. This is often used in geotechnical or engineering settings to get a quick estimate (minutes instead of days) of the aquifer properties immediately around the well.

Aquifer tests are typically interpreted by using an analytical model of aquifer flow (the most fundamental being the Theis solution) to match the data observed in the real world, then assuming that the parameters from the idealized model apply to the real-world aquifer. In more complex cases, a numerical model may be used to analyze the results of an aquifer test, but adding complexity does not ensure better results (see parsimony).

Aquifer testing differs from well testing in that the behaviour of the well is primarily of concern in the latter, while the characteristics of the aquifer are quantified in the former. Aquifer testing also often utilizes one or more monitoring wells, or piezometers ("point" observation wells). A monitoring well is simply a well which is not being pumped (but is used to monitor the hydraulic head in the aquifer). Typically monitoring and pumping wells are screened across the same aquifers.

....

Most commonly an aquifer test is conducted by pumping water from one well at a steady rate and for at least one day, while carefully measuring the water levels in the monitoring wells. When water is pumped from the pumping well the pressure in the aquifer that feeds that well declines. This decline in pressure will show up as drawdown (change in hydraulic head) in an observation well. Drawdown decreases with radial distance from the pumping well and drawdown increases with the length of time that the pumping continues.

The aquifer characteristics which are evaluated by most aquifer tests are:

Hydraulic conductivity The rate of flow of water through a unit cross sectional area of an aquifer, at a unit hydraulic gradient. In US units the rate of flow is in gallons per day per square foot of cross sectional area; in SI units hydraulic conductivity is usually quoted in m3 per day per m2. Units are frequently shortened to metres per day or equivalent.
Specific storage or storativity: a measure of the amount of water a confined aquifer will give up for a certain change in head;
Transmissivity The rate at which water is transmitted through whole thickness and unit width of an aquifer under a unit hydraulic gradient. It is equal to the hydraulic conductivity times the thickness of an aquifer;
Additional aquifer characteristics which are sometimes evaluated, depending on the type of aquifer, include:

Specific yield or drainable porosity: a measure of the amount of water an unconfined aquifer will give up when completely drained;
Leakage coefficient: some aquifers are bounded by aquitards which slowly give up water to the aquifer, providing additional water to reduce drawdown;
The presence of aquifer boundaries (recharge or no-flow) and their distance from the pumped well and piezometers.

\subsection{Beberapa kekurangan uji akuifer}

Alih-alih menyebutkan kelebihannya di awal, kami justru akan menyampaikan kekurangannya agar para analis dapat berhati-hati dalam menginterpretasi hasil uji akuifer.



\subsection{Sejarah uji akuifer}

Kami tidak akan menguraikan sejarah uji akuifer secara lengkap, namun salah satu uji aquifer terawal yang terdokumentasi lengkap akan kami ceritakan di sini, menyarikan uraian dari referensi \cite{kruseman1994analysis}.

...
...




\subsection{Asumsi dasar}

The Theis solution is based on the following assumptions:

\begin{itemize}
\item The flow in the aquifer is adequately described by Darcy's law (i.e. Re<10).
\end{itemize}
\begin{itemize}
\item homogeneous, isotropic, confined aquifer,
\end{itemize}
\begin{itemize}
\item well is fully penetrating (open to the entire thickness (b) of aquifer),
\end{itemize}
\begin{itemize}
\item the well has zero radius (it is approximated as a vertical line) — therefore no water can be stored in the well,
\end{itemize}
\begin{itemize}
\item the well has a constant pumping rate Q,
\end{itemize}
\begin{itemize}
\item the head loss over the well screen is negligible,
\end{itemize}
\begin{itemize}
\item aquifer is infinite in radial extent,
\end{itemize}
\begin{itemize}
\item horizontal (not sloping), flat, impermeable (non-leaky) top and bottom boundaries of aquifer,
\end{itemize}
\begin{itemize}
\item groundwater flow is horizontal
\end{itemize}
\begin{itemize}
\item no other wells or long term changes in regional water levels (all changes in potentiometric surface are the result of the pumping well alone)
\end{itemize}
\begin{itemize}
\end{itemize}

Even though these assumptions are rarely all met, depending on the degree to which they are violated (e.g., if the boundaries of the aquifer are well beyond the part of the aquifer which will be tested by the pumping test) the solution may still be useful.






\section{Alat dan bahan yang diperlukan}

Walaupun kami menyarankan pencatatan data dan analisisnya secara manual, namun tidak ada salahnya kami menyampaikan beberapa template spreadsheet yang biasa digunakan dalam uji akuifer.

\begin{itemize}
\item \href{https://pubs.usgs.gov/of/2002/ofr02197/}{Spreadsheets for the Analysis of Aquifer-Test and Slug-Test Data, Version 1.2 By Keith J. Halford and Eve L. Kuniansky (USGS)} 
\end{itemize}

...
...


\section{Berbagai uji akuifer}

Teknik rinci pengambilan data penurunan MAT serta pengolahan datanya akan sangat ditentukan oleh jenis pengujian yang dilakukan, misal apakah MAT diukur pada sumur pantau atau sekaligus diukur pada sumur produksi, apakah sumur menembus akuifer 100\% (\textit{full penetration}) atau tidak penuh (\textit{half penetration}). Variasi itu yang nantinya perlu diperhitungkan saat mengolah, menganalisis, dan menginterpretasi data. 

Pada buku ini, kami mengurutkan kondisi uji akuifer dari teknik yang paling sering dilakukan (tidak ideal) ke teknik yang ideal. Keputusan ini kami ambil agar para pembaca dapat membandingkan dan secara praktis dapat melakukan uji akuifer dengan biaya yang paling ekonomis, walaupun disadari hasilnya akan memiliki faktor kesalahan tertentu.


\subsection{Single well test (uji sumur tunggal)}

Terlepas dari kondisi ideal yang disyaratkan — salah satunya adalah pengukuran perubahan muka air tanah pada sumur pantau — uji sumur yang terbanyak dilakukan di Indonesia adalah uji sumur tunggal.  

