\subsection{Beberapa definisi}

\subsubsection{Uji akuifer}

\textbf{Uji akuifer} (atau uji pompa) dilaksanakan untuk mengevaluasi karakteristik akuifer dengan menstimulasi akuifer melalui pemompaan dan observasi terhadap respon aquifer berupa penurunan MAT. Pengujian ini juga umum dilakukan oleh hidrogeolog untuk mengkarakterisasi sistem akuifer, akuitar, pola aliran, dan batas aliran (\textit{flow system boundaries}) bila ada. 

Uji pemompaan yang biasa dilaksanakan dalam uji akuifer menggunakan debit konstan pada periode minimum satu hari (atau 24 jam), dengan mengukur MAT pada sumur pantau. Saat air dipompa ke permukaan, maka tekanan dalam akuifer akan menurun. Penurunan ini ditandai dengan penurunan MAT (atau \textit{hydraulic head}) pada sumur pantau (Gambar \ref{fig:pumping_diagram}). Besarnya penurunan ini akan berkurang dalam radius tertentu dari titik sumur pemompaan, radius ini dinamakan \textbf{radius pengaruh}. Sumur pompa dan sumur pantau memiliki saringan (\textit{screen}) pada akuifer yang sama. Berbagai tujuan pengujian dan disain pengujian dapat dilihat pada Tabel \ref{tab:tabel_pengujian}.

Karakter akuifer yang didapatkan dari pengujian ini mencakup: 

\begin{itemize}
    \item \textbf{Konduktivitas hidrolik} (Hydraulic conductivity): volume air yang mengalir melalui pada satu satuan luas akuifer per satuan gradien hidrolik. Satuannya galon/hari/m2 (dalam satuan US), dalam SI satuan yang digunakan adalah m3/hari/m2, atau disederhanakan menjadi m/hari (atau satuan yang relevan dengannya). Dinotasikan sebagai \verb|K|.
    \item \textbf{Storativitas} (\textit{storativity} atau \textit{specific storage}): jumlah air dalam akuifer tertekan yang mampu dikeluarkan per perubahan \textit{head}. Dinotasikan sebagai \verb|S|;
    \item \textbf{Transmisivitas} (\textit{transmissivity}): jumlah air yang mampu dialirkan untuk tiap satuan ketebalan dan lebar akuifer pada tiap unit gradien hidrolik. Dinotasikan sebagai \verb|T|. Dengan demikian bilangan ini akan mewakili jenis material dicerminkan oleh \verb|K| dan dimensi ketebalan \verb|b| akuifer;
    \item Karakter tambahan lainnya:
        \begin{itemize}
        \item \textbf{Debit efektif} (\textit{Specific yield}) atau porositas spesifik (\textit{drainable porosity}): Nilai jumlah air yang dilepaskan oleh akuifer tak tertekan saat dikeringkan sepenuhnya;
        \item \textbf{Koefisien bocoran} (\textit{Leakage coefficient}): Beberapa akuifer dibatasi oleh lapisan akuitar yang secara perlahan mengalirkan air ke lapisan akuifer lainnya;
        \item \textbf{Kehadiran batas akuifer} (\textit{aquifer boundaries}) dalam bentuk batas imbuhan (\textit{recharge boundary}) atau batas tanpa aliran (\textit{no flow boundary}) serta jaraknya dari sumur pompa dan sumur pantau.
        \end{itemize}
\end{itemize}

\begin{table} 
    \caption{Tujuan pengujian dan disain pengujian} 
    \label{tab:tabel_pengujian}
    \begin{tabular}{ c c }
        Tujuan pengujian & Disain pengujian \\ 
        Untuk menentukan debit pompa optimum & Uji sumur tunggal dengan debit bertingkat (\textit{Single well step-drawdown test}) \\ 
        Untuk menentukan radius pengaruh dampak pemompaan & Uji akuifer dengan beberapa sumur pantau (dengan radius ke sumur pemompaan berbeda) dengan debit pemompaan besar. \\ 
        Untuk memperkirakan dampak  terhadap air permukaan & Uji akuifer dengan beberapa sumur pantau dan bendung pada aliran air permukaan. \\ 
        Karakterisasi akuifer untuk investigasi sistem akuifer secara umum & Uji akuifer dengan waktu pemompaan tambahan pada akuifer yang berbeda-beda dalam radius dekat dengan sumur pemompaan \\ 
    \end{tabular} 
\end{table}

