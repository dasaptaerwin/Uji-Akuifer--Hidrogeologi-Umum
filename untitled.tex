\section{Status dokumen}

\begin{itemize}
\item Judul: Pumping test
\end{itemize}
\begin{itemize}
\item Penulis: Dasapta Erwin Irawan dan Deny Juanda Puradimaja
\end{itemize}
\begin{itemize}
\item Kontributor: Adhi Pramudito, ST dan Ali Lukman, ST
\end{itemize}
\begin{itemize}
\item Status: Merupakan bab tambahan dari Buku Hidrogeologi Umum Edisi 2016
\end{itemize}
\begin{itemize}
\item Lisensi: \href{http://creativecommons.org/licenses/by/2.0/}{CC-BY ver 2.0}
\end{itemize}

Berikut sedikit quote dari kontributor mengenai uji akuifer (atau disebut juga sebagai uji pompa) (Akun Instagram Adhi Pramudito).

\begin{quote}
Menyambung post sebelumnya, mungkin banyak yang bertanya-tanya \textit{ngapain} saya sok-sok an tidur di bawah rembulan? (haha kayak ada yang bertanya-tanya aja). Jawabannya adalah saya sedang melakukan \textit{pumping test}! 

Yup, lubang yang baru dibor wajib banget buat diuji kapasitasnya dengan cara \textit{pumping test} ini. Kenapa wajib ? Kita tahu bahwa keadaan air tanah di setiap daerah berbeda-beda, baik kapasitasnya maupun medianya. Kadang orang seenak jidat bilang saya mau debit 10 l/s tanpa tahu karakteristik geologinya seperti apa, yah kalau tidak ada hukum alam air mengalir di situ ya tidak akan ada airnya!

\textit{Pumping test} ini berguna untuk mengetahui kapasitas sebuah akifer air tanah, dari hasilnya nanti kita bisa tahu debit optimum yang bisa kita ambil dari akifer tersebut. Kalau hasilnya bilang 3 l/s, Insya Allah diambil berapa lama pun dengan debit segitu tidak akan merusak lingkungan. Yang sering terjadi adalah debit optimum hanya 3 l/s tapi dengan serakahnya orang ambil lebih dari itu, ya turunlah muka air tanahnya, ambles, terus semua marah-marah.

Apapun yang ada di alam kalau kita ambil lebih dari kapasitasnya pasti ujung-ujungnya berdampak buruk untuk manusianya sendiri, kalau alam sih cuma berusaha menyeimbangkan dirinya sendiri. Ya, makanya Allah selalu bilang kita tidak boleh berlebihan dalam hal apapun.
\end{quote}