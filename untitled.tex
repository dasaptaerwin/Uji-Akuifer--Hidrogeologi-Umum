Status dokumen

\begin{itemize}
\item Judul: Pumping test
\end{itemize}
\begin{itemize}
\item Penulis: Dasapta Erwin Irawan dan Deny Juanda Puradimaja
\end{itemize}
\begin{itemize}
\item Kontributor: Adhi Pramudito, ST dan Ali Lukman, ST
\end{itemize}
\begin{itemize}
\item Status: Merupakan bab tambahan dari Buku Hidrogeologi Umum Edisi 2016
\end{itemize}
\begin{itemize}
\item Lisensi: \href{http://creativecommons.org/licenses/by/2.0/}{CC-BY ver 2.0}
\end{itemize}


\section{Tujuan instruksional} 

Mahasiswa mampu memahami:
\begin{itemize}
\item Prinsip dasar uji pompa
\end{itemize}
\begin{itemize}
\item Bagaimana melakukan uji poma
\end{itemize}
\begin{itemize}
\item Mengambila data uji pompa
\end{itemize}


\section{Pendahuluan}

Uji pompa adalah pengujian properti hidrolik akuifer melalui proses pemompaan, dengan debit tetap maupun debit berubah. Teknik ini diyakini menghasilkan estimasi properti hidrolik akuifer yang paling tepat. 

Sejarah uji pompa


Asumsi dasar


Berbagai uji akuifer


\subsection{Single well test (uji sumur tunggal)}

Terlepas dari kondisi ideal yang disyaratkan — salah satunya adalah pengukuran perubahan muka air tanah pada sumur pantau — uji sumur yang terbanyak dilakukan di Indonesia adalah uji sumur tunggal.  
