\section{Status dokumen}

\begin{itemize}
\item Judul: Pumping test
\end{itemize}
\begin{itemize}
\item Penulis: Dasapta Erwin Irawan dan Deny Juanda Puradimaja
\end{itemize}
\begin{itemize}
\item Kontributor: Adhi Pramudito, ST dan Ali Lukman, ST
\end{itemize}
\begin{itemize}
\item Status: Merupakan bab tambahan dari Buku Hidrogeologi Umum Edisi 2016
\end{itemize}
\begin{itemize}
\item Lisensi: \href{http://creativecommons.org/licenses/by/2.0/}{CC-BY ver 2.0}
\end{itemize}

Berikut sedikit quote dari kontributor mengenai uji akuifer (atau disebut juga sebagai uji pompa) (Akun Instagram Adhi Pramudito).

\begin{quote}
Menyambung post sblmnya, mgkn banyak yg bertanya2 ngapain saya sok2an tdr di bawah rembulan ?(haha ky ad yg bertanya2 aja). Jawabannya adalah sy sedang melakukan pumping test! 

Yup, lubang yang baru di bor wajib banget buat diuji kapasitasnya dengan cara pumping test ini. Kenapa wajib ? Kita tahu bahwa keadaan airtanah di setiap daerah berbeda2, baik kapasitasnya maupun medianya. Kadang orang seenak jidat blg saya mau debit 10 l/s tanpa tau karakteristik geologinya seperti apa, yah klo ga ada hukum alam air ngalir disitu ya ga akan ada airnya!

Pumping test ini berguna untuk mengetahui kapasitas sebuah akifer airtanah, dari hasilnya nanti kita bisa tahu debit optimum yang bisa kita ambil dr akifer tersebut. Kalau hasilnya blg 3 l/s , InsyaAllah diambil berapa lama pun dengan debit segitu tidak akan merusak lingkungan. Yang sering terjadi adalah debit optimum cm 3 l/s tp dengan serakahnya org ambil lebih dr itu, ya turunlah muka airtanahnya, ambles, trs semua marah2.

Apapun yang ada di alam kalau kita ambil lebih dr kapasitasnya pasti ujung2nya berdampak buruk untuk manusianya sendiri, kalau alam sih cm berusaha menyeimbangkan dirinya sendiri. Ya, makanya Allah selalu blg kita tidak boleh berlebihan dalam hal apapun.
\end{quote}