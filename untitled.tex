Status dokumen

\begin{itemize}
\item Judul: Pumping test
\end{itemize}
\begin{itemize}
\item Penulis: Dasapta Erwin Irawan dan Deny Juanda Puradimaja
\end{itemize}
\begin{itemize}
\item Kontributor: Adhi Pramudito, ST dan Ali Lukman, ST
\end{itemize}
\begin{itemize}
\item Status: Merupakan bab tambahan dari Buku Hidrogeologi Umum Edisi 2016
\end{itemize}
\begin{itemize}
\item Lisensi: \href{http://creativecommons.org/licenses/by/2.0/}{CC-BY ver 2.0}
\end{itemize}


\section{Tujuan instruksional} 

Mahasiswa mampu memahami:
\begin{itemize}
\item Prinsip dasar uji pompa
\end{itemize}
\begin{itemize}
\item Bagaimana melakukan uji poma
\end{itemize}
\begin{itemize}
\item Mengambila data uji pompa
\end{itemize}


\section{Pendahuluan}

Uji pompa adalah pengujian properti hidrolik akuifer melalui proses pemompaan, dengan debit tetap maupun debit berubah. Teknik ini diyakini menghasilkan estimasi properti hidrolik akuifer yang paling tepat. 

\subsection{Sejarah uji pompa}



\subsection{Asumsi dasar}



\section{Alat dan bahan yang diperlukan}





\section{Berbagai uji akuifer}

Teknik rinci pengambilan data penurunan MAT serta pengolahan datanya akan sangat ditentukan oleh jenis pengujian yang dilakukan, misal apakah MAT diukur pada sumur pantau atau sekaligus diukur pada sumur produksi, apakah sumur menembus akuifer 100\% (\textit{full penetration}) atau tidak penuh (\textit{half penetration}). Variasi itu yang nantinya perlu diperhitungkan saat mengolah, menganalisis, dan menginterpretasi data. 

Pada buku ini, kami mengurutkan kondisi uji akuifer dari teknik yang paling sering dilakukan (tidak ideal) ke teknik yang ideal. Keputusan ini kami ambil agar para pembaca dapat membandingkan dan secara praktis dapat melakukan uji akuifer dengan biaya yang paling ekonomis, walaupun disadari hasilnya akan memiliki faktor kesalahan tertentu.


\subsection{Single well test (uji sumur tunggal)}

Terlepas dari kondisi ideal yang disyaratkan — salah satunya adalah pengukuran perubahan muka air tanah pada sumur pantau — uji sumur yang terbanyak dilakukan di Indonesia adalah uji sumur tunggal.  
