\section{Tahapan dan perlengkapan yang diperlukan}

\subsection{Tahap persiapan}

\begin{enumerate}
\item Perencanaan kegiatan: perencanaan secara rinci sangat diperlukan mengingat kegiatan ini memerlukan banyak peralatan, sumber daya manusia, dan biaya.
\item Peralatan yang diperlukan diantaranya: peralatan yang diperlukan dapat dibagi menjadi tiga kelompok besar. Alat pemboran (hanya diperlukan bila sumur belum dibuat), pompa selam (\textit{submersible pump}), instalasi listrik atau generator untuk menjalankan pompa selam, katup pengukuran debit pemompaan, \textit{water level detector} untuk mengukur posisi MAT saat pemompaan dilakukan, dan botol sample air berbahan plastik.

\item Perizinan: luangkan waktu untuk mengurus perizinan. Izin ini sangat penting karena kegiatan berpotensi mengganggu warga sekitar dari sisi suara yang bising, kotornya jalan akibat roda kendaraan, asap solar dari genset.

\item Perjalanan peralatan: perjalanan peralatan, terutama bila melibatkan truk pemboran, dapat menjadi petualangan yang menegangkan. Apalagi bila memerlukan perjalanan dengan ferry menyeberang selat dll. Perencanaan waktu perlu memperhitungkan hal-hal di luar dugaan, seperti jalan yang tergenang banjir, atau gelombang laut tinggi sehingga menghambat perjalanan kapal ferry, dll.

\item Pembersihan lahan (\textit{land clearance}):

\item


\item 

\end{enumerate}




\subsection{Tahap pemompaan}





\subsection{Tahap analisis}








\section{Teknis uji akuifer}






\section{Teknik uji sumur}




