\section{Pendahuluan}

Uji akuifer (UA) adalah pengujian properti hidrolik akuifer melalui proses pemompaan (uji pompa), dengan debit tetap maupun debit berubah. Teknik ini diyakini menghasilkan estimasi properti hidrolik akuifer yang paling tepat, serta perhitungan debit sumur optimum atau sering disebut debit aman (\textit{safe yield}). Dengan uji akuifer, kita juga akan mendapatkan profil penurunan MAT dalam proses pemompaan. Dalam jangka panjang, uji akuifer juga diharapkan dapat berkontribusi dalam menata lingkungan secara berkelanjutan. 


\subsection{Beberapa definisi}

\textbf{Uji akuifer} (atau uji pompa) dilaksanakan untuk mengevaluasi karakteristik akuifer dengan menstimulasi akuifer melalui pemompaan dan observasi terhadap respon aquifer berupa penurunan MAT. Pengujian ini juga umum dilakukan oleh hidrogeolog untuk mengkarakterisasi sistem akuifer, akuitar, pola aliran, dan batas aliran (\textit{flow system boundaries}) bila ada. 

Uji pemompaan yang biasa dilaksanakan dalam uji akuifer menggunakan debit konstan pada periode minimum satu hari (atau 24 jam), dengan mengukur MAT pada sumur pantau. Saat air dipompa ke permukaan, maka tekanan dalam akuifer akan menurun. Penurunan ini ditandai dengan penurunan MAT (atau \textit{hydraulic head}) pada sumur pantau. Besarnya penurunan ini akan berkurang dalam radius tertentu dari titik sumur pemompaan, radius ini dinamakan \textbf{radius pengaruh}. Sumur pompa dan sumur pantau memiliki saringan (\textit{screen}) pada akuifer yang sama. 

Karakter akuifer yang didapatkan dari pengujian ini mencakup: 

\begin{itemize}
    \item \textbf{Konduktivitas hidrolik} (Hydraulic conductivity): volume air yang mengalir melalui pada satu satuan luas akuifer per satuan gradien hidrolik. Satuannya galon/hari/m2 (dalam satuan US), dalam SI satuan yang digunakan adalah m3/hari/m2, atau disederhanakan menjadi m/hari (atau satuan yang relevan dengannya). Dinotasikan sebagai \verb|K|.
    \item \textbf{Storativitas} (\textit{storativity} atau \textit{specific storage}): jumlah air dalam akuifer tertekan yang mampu dikeluarkan per perubahan \textit{head}. Dinotasikan sebagai \verb|S|;
    \item \textbf{Transmisivitas} (\textit{transmissivity}): jumlah air yang mampu dialirkan untuk tiap satuan ketebalan dan lebar akuifer pada tiap unit gradien hidrolik. Dinotasikan sebagai \verb|T|. Dengan demikian bilangan ini akan mewakili jenis material dicerminkan oleh \verb|K| dan dimensi ketebalan \verb|b| akuifer;
    \item Karakter tambahan lainnya:
        \begin{itemize}
        \item Debit efektif (\textit{Specific yield}) atau porositas spesifik (\textit{drainable porosity}}: Nilai jumlah air yang dilepaskan oleh akuifer tak tertekan saat dikeringkan sepenuhnay;
        \item Koefisien bocoran (\textit{Leakage coefficient}): Beberapa akuifer dibatasi oleh lapisan akuitar yang secara perlahan mengalirkan air ke lapisan akuifer lainnya;
        \item \textbf{Kehadiran batas akuifer} (\textit{aquifer boundaries}) dalam bentuk batas imbuhan (\textit{recharge boundary}) atau batas tanpa aliran (\textit{no flow boundary}) serta jaraknya dari sumur pompa dan sumur pantau.
        \end{itemize}
\end{itemize}

\textbf{Uji sumur}, istilah ini sering disamakan dengan uji akuifer. Sebenarnya keduanya tidaklah sama. Bila uji akuifer dilaksanakan untuk menguji karakteristik akuifer, uji sumur dipakai untuk mengetahui karakteristik sumur. Pertanyaannya adalah, apakah karakter akuifer akan sama dengan karakter sumur? Jawabnya adalah tidak, karena sumur hanyalah satu titik yang mengeksploitasi air tanah yang mengalir pada suatu lapisan akuifer yang berdimensi panjang, lebar dan tebal. Disain konstruksi sumur yang buruk akan menyebabkan aliran air dari akuifer ke dalam sumur menjadi tidak lancar, dinyatakan sebagai \verb|efisiensi sumur| (\textit{well efficiency}).

\textbf{A slug test} adalah variasi uji akuifer yang mengamati perubahan MAT secara instan (baik peningkatan maupun penurunan) pada sumur yang sama (sumur produksi). Pengujian ini seringkali dilakukan dalam kegiatan pemetaan geoteknik untuk mendapatkan perkiraan cepat nilai hidrolik dalam skala waktu menit, bukan jam atau hari. \verb|Jelaskan lebih lengkap teknis slug test|.

Kedua pengujian ini diintepretasi dengan model analitis  aliran air tanah (the \textit{Theis solution}). Pada kondisi yang lebih kompleks model numerik bisa dipakai (\cite{lebbe1995validation}), namun perlu diingat bahwa model yang lebih rumit tidak menjamin hasil yang lebih baik (\cite{Johnson_2001}\cite{Rushton_1976}\cite{Rathod_1984}\cite{2011}\cite{Lebbe_1999}\cite{Lebbe_1999}). 

\subsection{Beberapa kekurangan uji akuifer}

Alih-alih menyebutkan kelebihannya di awal, kami justru akan menyampaikan kekurangannya agar para analis dapat berhati-hati dalam menginterpretasi hasil uji akuifer.



\subsection{Sejarah uji akuifer}

Kami tidak akan menguraikan sejarah uji akuifer secara lengkap, namun salah satu uji aquifer terawal yang terdokumentasi lengkap akan kami ceritakan di sini, menyarikan uraian dari referensi \cite{kruseman1994analysis}.

...
...




\subsection{Asumsi dasar}

Asumsi dasar Persamaan Theis untuk menganalisis data penurunan MAT:

\begin{itemize}
\item Aliran air dalam akuifer memiliki karakteristik sebagaimana dijelaskan dalam Hukum Darcy, misal: aliran homogeneous dan laminar. 
\item Akuifer bersifat homogen, isotropik, tertekan. 
\item Sumur menembus akuifer sepenuhnya (\textit{full penetration}) 
\item the well has zero radius (it is approximated as a vertical line) — therefore no water can be stored in the well,
\end{itemize}
\begin{itemize}
\item the well has a constant pumping rate Q,
\end{itemize}
\begin{itemize}
\item the head loss over the well screen is negligible,
\end{itemize}
\begin{itemize}
\item aquifer is infinite in radial extent,
\end{itemize}
\begin{itemize}
\item horizontal (not sloping), flat, impermeable (non-leaky) top and bottom boundaries of aquifer,
\end{itemize}
\begin{itemize}
\item groundwater flow is horizontal
\end{itemize}
\begin{itemize}
\item no other wells or long term changes in regional water levels (all changes in potentiometric surface are the result of the pumping well alone)
\end{itemize}
\begin{itemize}
\end{itemize}

Even though these assumptions are rarely all met, depending on the degree to which they are violated (e.g., if the boundaries of the aquifer are well beyond the part of the aquifer which will be tested by the pumping test) the solution may still be useful.

