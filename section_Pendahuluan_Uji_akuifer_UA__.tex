\section{Pendahuluan}

Uji akuifer (UA) adalah pengujian properti hidrolik akuifer melalui proses pemompaan (uji pompa), dengan debit tetap maupun debit berubah. Teknik ini diyakini menghasilkan estimasi properti hidrolik akuifer yang paling tepat, serta perhitungan debit sumur optimum atau sering disebut debit aman (\textit{safe yield}). Dengan uji akuifer, kita juga akan mendapatkan profil penurunan MAT dalam proses pemompaan. Dalam jangka panjang, uji akuifer juga diharapkan dapat berkontribusi dalam menata lingkungan secara berkelanjutan. 


\subsection{Beberapa definisi}

\textbf{Uji akuifer} (atau uji pompa) dilaksanakan untuk mengevaluasi karakteristik akuifer dengan menstimulasi akuifer melalui pemompaan dan observasi terhadap respon aquifer berupa penurunan MAT. Pengujian ini juga umum dilakukan oleh hidrogeolog untuk mengkarakterisasi sistem akuifer, akuitar, dan pola aliran. Aquifer testing is a common tool that hydrogeologists use to characterize a system of aquifers, aquitards and flow system boundaries.

\textbf{Uji sumur}, istilah ini sering disamakan dengan uji akuifer. Sebenarnya keduanya tidaklah sama. Bila uji akuifer dilaksanakan untuk menguji karakteristik akuifer, uji sumur dipakai untuk mengetahui karakteristik sumur. Pertanyaannya adalah, apakah karakter akuifer akan sama dengan karakter sumur? Jawabnya adalah tidak, karena sumur hanyalah satu titik yang mengeksploitasi air tanah yang mengalir pada suatu lapisan akuifer yang berdimensi panjang, lebar dan tebal. Disain konstruksi sumur yang buruk akan menyebabkan aliran air dari akuifer ke dalam sumur menjadi tidak lancar, dinyatakan sebagai \verb|efisiensi sumur| (\textit{well efficiency}).

\textbf{A slug test} adalah variasi uji akuifer yang mengamati perubahan MAT secara instan (baik peningkatan maupun penurunan) pada sumur yang sama (sumur produksi). Pengujian ini seringkali dilakukan dalam kegiatan pemetaan geoteknik untuk mendapatkan perkiraan cepat nilai hidrolik dalam skala waktu menit, bukan jam atau hari. 

Kedua pengujian ini diintepretasi dengan model analitis  aliran air tanah (the \textit{Theis solution}). Pada kondisi yang lebih kompleks model numerik bisa dipakai\cite{lebbe1995validation}, namun perlu diingat bahwa model yang lebih rumit tidak menjamin hasil yang lebih baik\cite{Johnson_2001}\cite{Rushton_1976}\cite{Rathod_1984}\cite{2011}\cite{Lebbe_1999}\cite{Lebbe_1999}. 

Anda mungkin sering bingung Aquifer testing differs from well testing in that the behaviour of the well is primarily of concern in the latter, while the characteristics of the aquifer are quantified in the former. Aquifer testing also often utilizes one or more monitoring wells, or piezometers ("point" observation wells). A monitoring well is simply a well which is not being pumped (but is used to monitor the hydraulic head in the aquifer). Typically monitoring and pumping wells are screened across the same aquifers.

Uji pemompaan yang biasa dilaksanakan dalam uji akuifer menggunakan debit konstan pada periode minimum satu hari (atau 24 jam), dengan mengukur MAT pada sumur pantau. Saat air dipompa ke permukaan, maka tekanan dalam akuifer akan menurun. Penurunan ini ditandai dengan penurunan MAT (atau \textit{hydraulic head}) pada sumur pantau. Besarnya penurunan ini akan berkurang dalam radius tertentu dari titik sumur pemompaan, radius ini dinamakan \textbf{radius pengaruh}. Drawdown decreases with radial distance from the pumping well and drawdown increases with the length of time that the pumping continues.

The aquifer characteristics which are evaluated by most aquifer tests are:

Hydraulic conductivity The rate of flow of water through a unit cross sectional area of an aquifer, at a unit hydraulic gradient. In US units the rate of flow is in gallons per day per square foot of cross sectional area; in SI units hydraulic conductivity is usually quoted in m3 per day per m2. Units are frequently shortened to metres per day or equivalent.
Specific storage or storativity: a measure of the amount of water a confined aquifer will give up for a certain change in head;
Transmissivity The rate at which water is transmitted through whole thickness and unit width of an aquifer under a unit hydraulic gradient. It is equal to the hydraulic conductivity times the thickness of an aquifer;
Additional aquifer characteristics which are sometimes evaluated, depending on the type of aquifer, include:

Specific yield or drainable porosity: a measure of the amount of water an unconfined aquifer will give up when completely drained;
Leakage coefficient: some aquifers are bounded by aquitards which slowly give up water to the aquifer, providing additional water to reduce drawdown;
The presence of aquifer boundaries (recharge or no-flow) and their distance from the pumped well and piezometers.

\subsection{Beberapa kekurangan uji akuifer}

Alih-alih menyebutkan kelebihannya di awal, kami justru akan menyampaikan kekurangannya agar para analis dapat berhati-hati dalam menginterpretasi hasil uji akuifer.



\subsection{Sejarah uji akuifer}

Kami tidak akan menguraikan sejarah uji akuifer secara lengkap, namun salah satu uji aquifer terawal yang terdokumentasi lengkap akan kami ceritakan di sini, menyarikan uraian dari referensi \cite{kruseman1994analysis}.

...
...




\subsection{Asumsi dasar}

Asumsi dasar Persamaan Theis untuk menganalisis data penurunan MAT:

\begin{itemize}
\item Aliran air dalam akuifer memiliki karakteristik sebagaimana dijelaskan dalam Hukum Darcy, misal: aliran homogeneous dan laminar. 
\item Akuifer bersifat homogen, isotropik, tertekan. 
\item Sumur menembus akuifer sepenuhnya (\textit{full penetration}) 
\item the well has zero radius (it is approximated as a vertical line) — therefore no water can be stored in the well,
\end{itemize}
\begin{itemize}
\item the well has a constant pumping rate Q,
\end{itemize}
\begin{itemize}
\item the head loss over the well screen is negligible,
\end{itemize}
\begin{itemize}
\item aquifer is infinite in radial extent,
\end{itemize}
\begin{itemize}
\item horizontal (not sloping), flat, impermeable (non-leaky) top and bottom boundaries of aquifer,
\end{itemize}
\begin{itemize}
\item groundwater flow is horizontal
\end{itemize}
\begin{itemize}
\item no other wells or long term changes in regional water levels (all changes in potentiometric surface are the result of the pumping well alone)
\end{itemize}
\begin{itemize}
\end{itemize}

Even though these assumptions are rarely all met, depending on the degree to which they are violated (e.g., if the boundaries of the aquifer are well beyond the part of the aquifer which will be tested by the pumping test) the solution may still be useful.

