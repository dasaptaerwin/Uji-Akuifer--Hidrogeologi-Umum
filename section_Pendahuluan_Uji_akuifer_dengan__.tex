\section{Pendahuluan}

Uji akuifer dengan pemompaan adalah salah satu teknologi yang tertua untuk mengetahui nilai parameter hidrolik. Salah satu uji aquifer terawal yang terdokumentasi lengkap akan kami ceritakan di sini, menyarikan uraian dari referensi \cite{kruseman1994analysis}.

Uji akuifer (UA) adalah pengujian properti hidrolik akuifer melalui proses pemompaan (uji pompa), dengan debit tetap maupun debit berubah. Teknik ini diyakini menghasilkan estimasi properti hidrolik akuifer yang paling tepat, serta perhitungan debit sumur optimum atau sering disebut debit aman (\textit{safe yield}). Dengan uji akuifer, kita juga akan mendapatkan profil penurunan MAT dalam proses pemompaan. Dalam jangka panjang, uji akuifer juga diharapkan dapat berkontribusi dalam menata lingkungan secara berkelanjutan. 

\subsection{Beberapa definisi}

\subsubsection{Uji akuifer}

\textbf{Uji akuifer} (atau uji pompa) dilaksanakan untuk mengevaluasi karakteristik akuifer dengan menstimulasi akuifer melalui pemompaan dan observasi terhadap respon aquifer berupa penurunan MAT. Pengujian ini juga umum dilakukan oleh hidrogeolog untuk mengkarakterisasi sistem akuifer, akuitar, pola aliran, dan batas aliran (\textit{flow system boundaries}) bila ada. 

Uji pemompaan yang biasa dilaksanakan dalam uji akuifer menggunakan debit konstan pada periode minimum satu hari (atau 24 jam), dengan mengukur MAT pada sumur pantau. Saat air dipompa ke permukaan, maka tekanan dalam akuifer akan menurun. Penurunan ini ditandai dengan penurunan MAT (atau \textit{hydraulic head}) pada sumur pantau (Gambar \ref{fig:pumping_diagram}). Besarnya penurunan ini akan berkurang dalam radius tertentu dari titik sumur pemompaan, radius ini dinamakan \textbf{radius pengaruh}. Sumur pompa dan sumur pantau memiliki saringan (\textit{screen}) pada akuifer yang sama. Berbagai tujuan pengujian dan disain pengujian dapat dilihat pada Tabel \ref{tab:tabel_pengujian}.

Karakter akuifer yang didapatkan dari pengujian ini mencakup: 

\begin{itemize}
    \item \textbf{Konduktivitas hidrolik} (Hydraulic conductivity): volume air yang mengalir melalui pada satu satuan luas akuifer per satuan gradien hidrolik. Satuannya galon/hari/m2 (dalam satuan US), dalam SI satuan yang digunakan adalah m3/hari/m2, atau disederhanakan menjadi m/hari (atau satuan yang relevan dengannya). Dinotasikan sebagai \verb|K|.
    \item \textbf{Storativitas} (\textit{storativity} atau \textit{specific storage}): jumlah air dalam akuifer tertekan yang mampu dikeluarkan per perubahan \textit{head}. Dinotasikan sebagai \verb|S|;
    \item \textbf{Transmisivitas} (\textit{transmissivity}): jumlah air yang mampu dialirkan untuk tiap satuan ketebalan dan lebar akuifer pada tiap unit gradien hidrolik. Dinotasikan sebagai \verb|T|. Dengan demikian bilangan ini akan mewakili jenis material dicerminkan oleh \verb|K| dan dimensi ketebalan \verb|b| akuifer;
    \item Karakter tambahan lainnya:
        \begin{itemize}
        \item \textbf{Debit efektif} (\textit{Specific yield}) atau porositas spesifik (\textit{drainable porosity}): Nilai jumlah air yang dilepaskan oleh akuifer tak tertekan saat dikeringkan sepenuhnya;
        \item \textbf{Koefisien bocoran} (\textit{Leakage coefficient}): Beberapa akuifer dibatasi oleh lapisan akuitar yang secara perlahan mengalirkan air ke lapisan akuifer lainnya;
        \item \textbf{Kehadiran batas akuifer} (\textit{aquifer boundaries}) dalam bentuk batas imbuhan (\textit{recharge boundary}) atau batas tanpa aliran (\textit{no flow boundary}) serta jaraknya dari sumur pompa dan sumur pantau.
        \end{itemize}
\end{itemize}

\begin{table} 
    \caption{Tujuan pengujian dan disain pengujian \cite{aitchison2008aquifer} }
    \label{tab:tabel_pengujian}
    \begin{tabular}{ c c }
        Tujuan pengujian & Disain pengujian \\ 
        Untuk menentukan debit pompa optimum & Uji sumur tunggal dengan debit bertingkat (\textit{Single well step-drawdown test}) \\ 
        Untuk menentukan radius pengaruh dampak pemompaan & Uji akuifer dengan beberapa sumur pantau (dengan radius ke sumur pemompaan berbeda) dengan debit pemompaan besar. \\ 
        Untuk memperkirakan dampak  terhadap air permukaan & Uji akuifer dengan beberapa sumur pantau dan bendung pada aliran air permukaan. \\ 
        Karakterisasi akuifer untuk investigasi sistem akuifer secara umum & Uji akuifer dengan waktu pemompaan tambahan pada akuifer yang berbeda-beda dalam radius dekat dengan sumur pemompaan \\ 
    \end{tabular} 
\end{table}

\subsubsection{Uji sumur}

\textbf{Uji sumur}, istilah ini sering disamakan dengan uji akuifer. Sebenarnya keduanya tidaklah sama. Bila uji akuifer dilaksanakan untuk menguji karakteristik akuifer, uji sumur dipakai untuk mengetahui karakteristik sumur. Pertanyaannya adalah, apakah karakter akuifer akan sama dengan karakter sumur? Jawabnya adalah tidak, karena sumur hanyalah satu titik yang mengeksploitasi air tanah yang mengalir pada suatu lapisan akuifer yang berdimensi panjang, lebar dan tebal. Disain konstruksi sumur yang buruk akan menyebabkan aliran air dari akuifer ke dalam sumur menjadi tidak lancar, dinyatakan sebagai \verb|efisiensi sumur| (\textit{well efficiency}). Gambar \ref{fig:drilling_truck} memperlihatkan tipikal \textit{truck mounted drilling rig}.

\subsubsection{Slug test}

\textbf{A slug test} adalah variasi uji akuifer yang mengamati perubahan MAT secara instan (baik peningkatan maupun penurunan) pada sumur yang sama (sumur produksi). Pengujian ini seringkali dilakukan dalam kegiatan pemetaan geoteknik untuk mendapatkan perkiraan cepat nilai hidrolik dalam skala waktu menit, bukan jam atau hari. Benda yang disebut \textbf{slug} sendiri adalah sebatang besi yang dijatuhkan ke dalam sumur agar muka air tanah berubah, naik untuk kemudian turun kembali ke posisi semula. Perubahan tersebut diukur, sehingga didapatkan kondisi seperti ada pompa di dalam sumur tersebut (Gambar \ref{fig:slugtest}).

Beberapa pengujian di atas diintepretasi dengan model analitis  aliran air tanah (the \textit{Theis solution}). Pada kondisi yang lebih kompleks model numerik bisa dipakai (\cite{lebbe1995validation}), namun perlu diingat bahwa model yang lebih rumit tidak menjamin hasil yang lebih baik (\cite{Johnson_2001}\cite{Rushton_1976}\cite{Rathod_1984}\cite{2011}\cite{Lebbe_1999}\cite{Lebbe_1999}). 

\subsection{Beberapa kekurangan uji akuifer}

Alih-alih menyebutkan kelebihannya di awal, kami justru akan menyampaikan kekurangannya agar para analis dapat berhati-hati dalam menginterpretasi hasil uji akuifer. Kekurangan yang jelas terlihat adalah ia memerlukan banyak asumsi, seperti dapat dilihat pada bagian di bawah ini.

Satu kekurangan yang terlihat adalah bahwa perhitungan uji pompa saat ini lebih menonjolkan aliran horizontal. Aliran vertikal tidak mendapatkan porsi yang cukup. Oleh karenanya bila ada distribusi nilai \verb|K| yang berpotensi menyebabkan aliran vertikal, maka tidak akan muncul dalam hasil pengujian, walaupun dapat menganggu nilainya.


\subsection{Asumsi dasar}

Asumsi dasar Persamaan Theis untuk menganalisis data penurunan MAT:

\begin{itemize}
\item Aliran air dalam akuifer memiliki karakteristik sebagaimana dijelaskan dalam Hukum Darcy, misal: aliran homogeneous dan laminar, 
\item Akuifer bersifat homogen, isotropik, tertekan, lapisan akuifer yang dipompa memiliki pelamparan lateral yang tidak terbatas, horizontal (tidak memiliki kemiringan), dibatasi oleh lapisan impermeabel (\textit{non-leaky}) pada bagian atas dan bawah akuifer,
\item Sumur menembus akuifer sepenuhnya (\textit{full penetration}), memiliki diameter 0 atau diasumsikan sebagai satu garis vertikal yang tidak memiliki diameter, sehingga tidak ada air yang tersimpan di dalamnya,
\item Debit pemompaan konstan sebesar \verb|Q|,
\item Perbedaan muka air (\textit{head loss}) di area sekitar saringan sumur (\textit{well screen}) dianggap tidak ada,
\item Tidak ada sumur lain yang berada di sekitar lokasi yang dapat mempengaruhi posisi MAT, serta tidak ada obyek lainnya di permukaan tanah yang dapat mempengaruhi pengukuran posisi air tanah, misal: rel kereta api, jalan raya, dll.
\end{itemize}

Disadari bahwa asumsi-asumsi di atas jarang dapat dijumpai, namun pendekatan yang diberikan masih dapat digunakan. 

